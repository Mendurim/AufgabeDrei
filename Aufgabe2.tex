\documentclass[10pt]{article}

\title{CS102 \LaTeX ~\"Ubung}

\author{Carlos Mojentale}

\date{\today}

\begin{document}

\maketitle

\section{Das ist das erste Abschnitt}

Hier k\"onnte auch anderer Text stehen.

\section{Tabelle}

Unsere wichtigsten Daten finden Sie in Tabelle 1.

\begin{table}[!th]

\centering

\begin{tabular}{c|c|c|c}

& Punkte erhalten & Punkte m\"oglich & \% \\

\hline

Aufgabe 1 & 2 & 4 & 0.5 \\

Aufgabe 2 & 3 & 3 & 1 \\

Aufgabe 3 & 3 & 3 & 1 \\

\end{tabular}

\caption{Diese Tabelle kann auch andere Werte beinhalten.}

\label{table:diese Tabelle kann auch andere Werte beinhalten.}

\end{table}

\section{Formeln}

\subsection{Pythagoras}

Der Satz der Pythagoras errechnet sich wie folgt: $a^{2} + b^{2} =\ c^{2}$. Daraus k\"onnen 
\\wir die Lange der Hypotenuse wie folgt berechnen: $c =\sqrt{a^{2} + b^{2}}$ 

\subsection{Summen}

Wir k\"onnen auch die Formel f\"ur eine Summe angeben:

\begin{equation}
s = \sum_{i=1}^n i = \frac{n*(n+1)}{2}
\end{equation}

\end{document}






